\newglossaryentry{Alt}{
	name={Alt},
	description={the Alt protein is ejected by the phage into the cytoplasm, where it modifies host RNA polymerase}
}
\newglossaryentry{Caudovirales}{
	name={\textit{Caudovirales}},
	description={Taxon of the tailed phages (Myophages, Siphophages, Podophages)}
}
\newglossaryentry{E. coli B}{
	name={\bact{ecolib}},
	description={classic non-pathogenic laboratory strain of \bact{ecoli}; source was J. Bordet. Mandated for use with the T series (Seven Dwarves) in the Phage Course.}
}
\newglossaryentry{Felix d'Herelle}{
	name={Felix d'Herelle},
	description={founder of phage biology. See Module 1.}
}
\newglossaryentry{Holliday junction}{
	name={Holliday junction},
	description={the interchange junction of DNA strands during recombination}
}
\newglossaryentry{Okazaki fragment}{
	name={Okazaki fragment},
	description={the DNA fragments synthesized on the lagging strand during replication}
}
\newglossaryentry{SD}{
	name={Shine-Dalgarno},
	description={``The Shine-Dalgarno (SD) sequence is a ribosomal binding site in prokaryotic messenger RNA.'' The canonical sequence is \texttt{AGGAGGU}\url{https://en.wikipedia.org/wiki/Shine-Dalgarno_sequence}}
}
\newglossaryentry{SOS response}{
	name={SOS response},
	description={DNA damage results in this mechanism triggering a lysogen to undergo the lytic cycle}
}
\newglossaryentry{Wac}{
	name={Wac},
	description={the Wac protein forms the phage whiskers and collar structures}
}
\newglossaryentry{abortive infection (AI)}{
	name={abortive infection (AI)},
	description={AI is a strategy by which bacteria block the propagation of phage by committing suicide upon infection. Usually the AI gene or genes are encoded on a prophage but, unlike other prophage genes, are constitutively expressed. AI generally is manifested by a sudden lethal membrane event at some time after the beginning of infection of a virulent phage, before the accumulation of progeny. The molecular basis of AI events is unknown. The best studied AI system is the rex system encoded by phage lambda.}
}
\newglossaryentry{activator}{
	name={activator},
	description={Also transcription factor. A protein that causes the activation of a cryptic promoter, either by binding to the DNA or to RNA polymerase.}
}
\newglossaryentry{adaptive immunity}{
	name={adaptive immunity},
	description={``Adaptive immunity creates immunological memory after an initial response to a specific pathogen, leads to an enhanced response to subsequent encounters with that pathogen. This process of acquired immunity is the basis of vaccination. Like the innate system, the adaptive system includes both humoral immunity components and cell-mediated immunity components.''\cite{wiki:Adaptive_immune_system}}
}
\newglossaryentry{annotation}{
	name={annotation},
	description={Annotation is the process of describing the contents of a genome. M. Zorn has a nice definition: Extraction, definition, and interpretation of features on the genome sequence derived by integrating computational tools and biological knowledge. (). The biggest task is finding the genes and assigning functional identities to the predicted gene products.}
}
\newglossaryentry{anti-terminator}{
	name={anti-terminator},
	description={A factor that reduces or abolishes transcriptional termination. The first examples were the N and Q proteins of phage lambda. However, in some cases, the anti-termination factor is a cis-acting RNA sequence.}
}
\newglossaryentry{att}{
	name={att},
	description={Also attB, attP, attachment sites. DNA sites on the phage DNA (attP) or bacterial chromosome (attB) that are used as substrates by phage-encoded Integrase for site-specific recombination. Result is integration of the phage DNA into the host chromosome.}
}
\newglossaryentry{attB}{
	name={attB},
	description={a site of \bact{ecoli} DNA necessary for recombination}
}
\newglossaryentry{attP}{
	name={attP},
	description={a site on lambda DNA necessary for recombination}
}
\newglossaryentry{baseplate}{
	name={baseplate},
	description={the apparatus at the bottom of a myophage tail to which the tail fibers are attached. Baseplates undergo major reorganization to cause sheath contraction and allow the tail tube to penetrate to the cell membrane surface. }
}
\newglossaryentry{beta-helix}{
	name={beta-helix},
	description={a protein structural motif}
}
\newglossaryentry{burst size}{
	name={burst size},
	description={Average pfu per infective center; usually determined by single-step growth experiment}
}
\newglossaryentry{bushing}{
	name={bushing},
	description={A term applied to a 10-membered ring of the lambda MCP, gpE, in contact with the portal. These E proteins are modified by covalent exchange reactions with C proteins, generating X1 and X2 products. The bushing concept reflects some expectation that the portal ring may turn with respect to the capsid, as a consequence between the symmetry mismatch at the portal vertex (5 fold axis at each vertex but 12-fold rotational symmetry for the portal.)}
}
\newglossaryentry{capsomer}{
	name={capsomer},
	description={the structural unit of virus head capsid}
}
\newglossaryentry{chaperonin}{
	name={chaperonin},
	description={multi-meric complex that aids in protein folding}
}
\newglossaryentry{clear plaques}{
	name={clear plaques},
	description={hallmark of virulent phages; plaques lack visible bacterial growth}
}
\newglossaryentry{cleared lysate}{
	name={cleared lysate},
	description={Crude product of phage infection or induction; cleared of large cell debris and usually stored over chloroform in the cold.}
}
\newglossaryentry{cohesive ends}{
	name={cohesive ends},
	description={the region of unpaired nucleotides on each end of lambda DNA prior to circularization}
}
\newglossaryentry{coiled coil}{
	name={coiled coil},
	description={a structural motif where multiple copies of a protein helix are coiled into a multimeric helix}
}
\newglossaryentry{coliphage}{
	name={coliphage},
	description={phage of \bact{ecoli}}
}
\newglossaryentry{constitutive}{
	name={constitutive},
	description={a promoter that is consistently active in the cell}
}
\newglossaryentry{contractile tail}{
	name={contractile tail},
	description={tail of a myophage}
}
\newglossaryentry{cos}{
	name={cos},
	description={an abbreviation for the cohesive ends of lambda DNA}
}
\newglossaryentry{coverage}{
	name={coverage},
	description={The percentage of a genome which is occupied by coding sequences. There are numerous methods for calculating coverage, one of the common (but wrong) options is to add up the lengths of all coding sequences and divide by the length of the genome. Consider why this might be wrong for phages, whereas it might work well for other organisms.}
}
\newglossaryentry{cryptic}{
	name={cryptic},
	description={a cryptic promoter has low or no functionality without transcription factors}
}
\newglossaryentry{cubic}{
	name={cubic},
	description={symmetry term that includes icosahedral symmetry}
}
\newglossaryentry{defective prophage}{
	name={defective prophage},
	description={prophages that have undergone mutations, including deletions, such that a viable virion is no longer encoded. \bact{ecolik12} has several, including dlp at \SI{12}{\min}, for example.}
}
\newglossaryentry{deletionogenic event}{
	name={deletionogenic event},
	description={loss of DNA material during replication}
}
\newglossaryentry{domain}{
	name={domain},
	description={a structural unit of a protein}
}
\newglossaryentry{efficiency of plating}{
	name={efficiency of plating},
	description={usually the relative pfu titer on one strain compared to another. Absolute EOP is pfu per virion. This is 1 for T4 and lambda.}
}
\newglossaryentry{endolysin}{
	name={endolysin},
	description={enzyme that cleaves host cell wall during the lytic cycle}
}
\newglossaryentry{establishment}{
	name={establishment},
	description={an established promoter is in its active form}
}
\newglossaryentry{h mutant}{
	name={\textit{h} mutant},
	description={short for host range, a mutant capable of infecting bacteria species other than its normal host}
}
\newglossaryentry{helix-turn-helix}{
	name={helix-turn-helix},
	description={a protein structural motif}
}
\newglossaryentry{hetero-immune}{
	name={hetero-immune},
	description={If a host cell, usually a lysogen, expresses a different, incompatible lysogenic repressor than an incoming temperate phage, this is considered to be a hetero-immune infection and, in general, results in a normal infection cycle for the incoming phage and also turbid plaque formation. Contrast with homo-immune.}
}
\newglossaryentry{holin}{
	name={holin},
	description={protein which creates holes in the IM, allowing endolysins to be released into the cytoplasm}
}
\newglossaryentry{homo-immune}{
	name={homo-immune},
	description={A bacterial strain that expresses a repressor compatible (capable of repressing) a particular phage is considered to be homo-immune to that phage. Usually this means it carries the prophage form of the phage, but in some cases, it might carry only a fragment of the prophage or even a different prophage that just happens to have a compatible repressor. In general, homo-immunity blocks gene expression in the incoming phage, leaving it as a non-replicating, repressed DNA molecule. In general, homo-immunity blocks plaque formation. See immunity.}
}
\newglossaryentry{homolog}{
	name={homolog},
	description={a homologous protein has similar DNA sequence to a protein in a different organism}
}
\newglossaryentry{homology-dependent recombination}{
	name={homology-dependent recombination},
	description={the universally-conserved process by which dsDNA molecules with regions of homologous undergo breakage and reunion recombination.  In E. coli, the key protein required for all homologous recombination is RecA. }
}
\newglossaryentry{hyper-recombinational}{
	name={hyper-recombinational},
	description={the state of elevated recombination attained in cells after DNA damage during the SOS response and also during phage infections.  Phages in general induce the hyper-rec state by expressing their own recombination proteins.  e.g., the lambda Red proteins.}
}
\newglossaryentry{immunity}{
	name={immunity},
	description={A bacterial strain is said to be immune to a temperate phage if the strain expresses a compatible lysogenic repressor that represses the phage upon entry of the DNA into the cytosol. Usually, this means that the immune host is already a lysogen of the phage. The lysogenic repressor was often referred to as the ``immunity'' substance. Unfortunate recent terminology has also brought in the CRISPR defense under the umbrella term of immunity, but it should be considered a form of insensitivity.}
}
\newglossaryentry{induction}{
	name={induction},
	description={the process of phage DNA exiting the lysogenic state and undergoing the lytic cycle}
}
\newglossaryentry{insensitivity}{
	name={insensitivity},
	description={A bacterial strain is considered to be insensitive to a phage if the phage cannot propagate on the strain under the specified conditions. Since bacteria have been ``at war'' with phages for a long time, bacteria have developed many ways to interfere with phage predation. A common form of insensitivity is true resistance. However, many other host defences can lead to complete or partial insensitivity, including restriction, CRISPR acquired immunity, abortive infection strategies, and other phenomena. It is important to note that a cell can be killed by a phage even if it is insensitive to the phage. Sensitivity refers to the ability of the phage to propagate, rather than the ability of the cell to survive. In some cases of insensitivity, the phage is able to eject its DNA into the cell and the cell dies, but phage progeny are not produced, at least not enough to cause propagation of the phages in the particular growth condition.}
}
\newglossaryentry{inverted repeat}{
	name={inverted repeat},
	description={an inverted DNA sequence found in operators}
}
\newglossaryentry{juggernaut}{
	name={juggernaut},
	description={the juggernaut refers to the complex of N protein, RNA, and RNA polymerase. This complex does not allow termination as it moves forward transcribing RNA.}
}
\newglossaryentry{linear concatemers}{
	name={linear concatemers},
	description={Most phages form linear concatemers of their unit genome as substrates for Terminase-dependent DNA packaging. In lambda, linear concatemers of the phage genome are formed by rolling circle replication and enhanced in terms of length by Red-mediated recombination. }
}
\newglossaryentry{lipobox}{
	name={lipobox},
	description={a four residue motif found in signal sequences of bacterial lipoproteins.  The motif is {L,V,I}-X-{G,S,A)-C, where X is never charged and rarely hydrophilic.  Signal peptidase II cleaves the preprotein in front of the Cys residue, leading to covalent lipoylation of the Cys on the free amino terminus and the sulfur atom.  }
}
\newglossaryentry{lysis}{
	name={lysis},
	description={the sudden loss of turbidity (cell density) in a bacterial culture, usually accompanied by a dramatic increase in viscosity, due to the release of host and phage DNA.  At the single cell level, the cell undergoes disruption of the envelope, resulting in the release of cytosolic contents and loss of phase contrast refractility. For Caudovirales, lysis at the single cell level is explosive.}
}
\newglossaryentry{lysogenic repressor}{
	name={lysogenic repressor},
	description={an example in phage biology would be CI protein which, by repressing competing proteins, causes the pathway to proceed lysogenically}
}
\newglossaryentry{lysozyme}{
	name={lysozyme},
	description={enzyme which catalyzes degredation of the bacterial cell wall}
}
\newglossaryentry{membrane potential}{
	name={membrane potential},
	description={the voltage across the inner membrane of bacterial cells}
}
\newglossaryentry{moron}{
	name={moron},
	description={a DNA element that is found in a specific region on the chromosome, but not in all phage particles}
}
\newglossaryentry{myophage}{
	name={myophage},
	description={a phage type charactarized by a contractile tail}
}
\newglossaryentry{nicks}{
	name={nicks},
	description={refers to a small single-stranded region in dsDNA, usually only a few nucleotides}
}
\newglossaryentry{noisy}{
	name={noisy},
	description={Noisy here describes a low signal-to-noise ratio, i.e. a lot of noise and very little signal.}
}
\newglossaryentry{operator}{
	name={operator},
	description={a region of DNA that binds transcription factors and regulates gene expression}
}
\newglossaryentry{permuted}{
	name={permuted},
	description={a phage genome where no defined start or stop is present}
}
\newglossaryentry{pinholin}{
	name={pinholin},
	description={pinholins create small gaps in the host IM, similar to holins}
}
\newglossaryentry{plaque}{
	name={plaque},
	description={A circular zone with reduced turbidity in a bacterial lawn. Each plaque is caused by a single phage particle that has successfully infected a cell and caused release of progeny, which then go on to diffuse in the agar to infect surrounding cells. For Caudovirales, plaques typically contain \textasciitilde\num{d6} virus particles. Temperate phages can form turbid plaques, because some cells become lysogenized by the phage and thus acquire immunity, allowing the lysogenic cells to grow within the plaque. Plaques can be completely clear, usually indicating that the phage is not capable of lysogenizing the host. Used for forming the lawn.}
}
\newglossaryentry{podophage}{
	name={podophage},
	description={a phage type charactarized by a short tail}
}
\newglossaryentry{positive control}{
	name={positive control},
	description={Positive control of gene expression means that in the absence of the regulatory molecule, expression is abrogated or reduced.  Positive control factors for transcription in phage biology are in two forms: anti-terminators (e.g., lambda N and Q),  and transcription factors (i.e., like lambda CII) which bind to DNA and activate transcription.  }
}
\newglossaryentry{prohead}{
	name={prohead},
	description={a mature prohead is the intermediate that is recognized by the Terminase to begin DNA packaging. }
}
\newglossaryentry{prolate}{
	name={prolate},
	description={charactarizes elongation in the vertical direction}
}
\newglossaryentry{promoter}{
	name={promoter},
	description={a promoter is a DNA locus that causes RNA polymerase to initiate transcription at a specific nearby site.}
}
\newglossaryentry{prophage}{
	name={prophage},
	description={A prophage is a dormant form of a temperate phage. A prophage is kept in its inactive state by a lysogenic repressor encoded on the prophage. Most prophages are integrated as a single copy into the host chromosome. However, some temperate phages form plasmid prophages. On average, bacterial genomes have been found to contain 3-6 recognizable prophage elements and often many more fragments of prophages, often designated as defective prophages. In truth however, it is not known for most prophage elements whether or not the element encodes a viable phage or not. All it would take is a single base pair defect in an essential gene and the prophage would be defective. Only when we can isolate the phage particle from a culture of the lysogenic cell can we be sure the prophage is functional.}
}
\newglossaryentry{pseudogene}{
	name={pseudogene},
	description={a gene which no longer functions due to a mutation}
}
\newglossaryentry{quasi-equivalence}{
	name={quasi-equivalence},
	description={For all T\textgreater{}1 capsid structures, the MCP assumes T slightly different (quasi-equivalent) structures. Thus in a T=7 phage like lambda, the MCP is in 7 slightly different conformations, one of which is the vertex pentameric structure, and the other 6 are structures found in different hexamers. The Caspar-Klug theory of quasi-equivalence explained how it could be that viruses can have vastly different capsid sizes, yet all have about the same size MCP. The alternative would have been that every virus had T=1 symmetry and be made up of 12 pentamers, so to get bigger (and be able to encapsulate more nucleic acid), the MCP itself would have to get bigger in all three dimensions during evolution. }
}
\newglossaryentry{regulon}{
	name={regulon},
	description={a series of genes that are similarly regulated}
}
\newglossaryentry{replisome}{
	name={replisome},
	description={the multi-protein complex that functions to replicate DNA}
}
\newglossaryentry{repressor}{
	name={repressor},
	description={a protein that binds to an operator and inhibits transcription of that gene}
}
\newglossaryentry{resistant}{
	name={resistant},
	description={A bacterial strain exhibits resistance to a particular phage if the phage is unable to adsorb to the host cell and transfer its nucleic acid into the cytosol. The most common form of resistance is the loss or mutational alteration of the receptor on the surface of the host.}
}
\newglossaryentry{restriction}{
	name={restriction},
	description={In phage biology, a phage is said to be restricted on a host if the bacterium carries a restriction enzyme system that can cleave the phage DNA. Restriction enzymes recognize a specific sequence in the phage DNA and, in general, cleave the DNA if it is not chemically modified in a certain way recognized by the restriction enzyme. Restriction systems always include an enzyme activity that will modify DNA in the host, so the incoming foreign DNA can be recognized and attacked. In general, efficient restriction systems prevent killing of the infected cell, since the phage DNA is degraded before it can be expressed. However, since the modification activity is also around, the ability to recognize the phage DNA and attack it represents a competion between the restriction cleavage and modification processes. Typically, in about one of a d-2d-4infections, the phage DNA gets modified before it is cleaved. In this case, the modified phage is then able to replicate, cause lysis and release modified progeny, which no longer are attenuated by the surrounding host cells. Thus modification can be considered a form of epigenetic change; the DNA sequence of the phage has not changed but it has acquired the ability to grow on the restricting host. Often, this means that the modified phage will be recognized by the restriction system of the host on which the phage was originally grown, so it will now be restricted on that host.}
}
\newglossaryentry{sensitivity}{
	name={sensitivity},
	description={A bacterial strain is considered to be sensitive to a phage if the phage can propagate in a culture of those cells. Obviously, the ability to propagate depends on a number of factors and can range from weak to very robust. Usually, sensitivity means ability to form a plaque on the lawn of the sensitive cells. However, there are some conditions where plaque formation in an agar lawn does not reflect the ability of the phage to propagate. For example, extremely large phage particles might not be able to diffuse well enough to form a plaque. In some cases, agar plating conditions lead to physiological changes that abrogate the ability of a particular phage to form plaques, yet in liquid culture the phage can propagate and grow to reasonable or high titers and cause lysis. Contrast to insensitivity.}
}
\newglossaryentry{siphophage}{
	name={siphophage},
	description={a phage type charactarized by a long, noncontractile tail}
}
\newglossaryentry{site-specific recombination}{
	name={site-specific recombination},
	description={Site-specific recombination is a recombination event that does not rely on sequence homology but instead occurs between specific sites and requires specific proteins other than the Rec system.  The most famous example of site-specific recombination in phage biology is the reaction between attP in the lambda DNA and attB in the \bact{ecoli} chromosome, catalyzed by the phage Int protein and resulting in the integration of the lambda DNA as a prophage.}
}
\newglossaryentry{spanin}{
	name={spanin},
	description={functions in aiding the OM disruption during lysis}
}
\newglossaryentry{specificity}{
	name={specificity},
	description={Phages usually show specificity for strains that have the receptor recognized by the tail fibers or other adsorption device of the virion. Some phages are specific for only a few strains of a bacterial species; this is also referred to as having a narrow host range. This in general is due to the receptor being a particular sugar structure or protein on the cell surface. If this receptor is not found in most strains of the bacterial species, the phage would be considered highly specific. Although the general opinion in modern phage therapy is that phages are highly specific anti-bacterial agents, some phages show very broad host range by recognizing universally conserved receptors. Note that specificity can be negatively over-ridden by other factors including restriction systems, etc.}
}
\newglossaryentry{specificity loop}{
	name={specificity loop},
	description={this is a small N-terminal domain of the ``single-subunit'' RNA polymerases, the prototype for which is gp\textit{1} of T7 phage.  This loop inspects the duplex DNA as it enters the active site and looks for the long promoter sequence. }
}
\newglossaryentry{sub-operators}{
	name={sub-operators},
	description={an operator within a tripartite operator}
}
\newglossaryentry{symmetry}{
	name={symmetry},
	description={In crystallographic terminology, an N-fold symmetry axis exists if a structure can be rotated in N steps without changing the structure. So a 2-fold symmetry axis can be rotated \ang{180} and look the same. In phage biology, the phage head or capsid has multiple symmetry axes. Each of the icosahedral vertices has a 5-fold axis in its middle. In all heads with T\textgreater{}1, there are hexamers of the MCP with 2-, 3- and 6-fold axes of symmetry. At the portal vertex in \textit{Caudovirales}, the portal has 12 subunits and 2-, 3-, 4-, 6- and 12-fold axes in the middle of the portal. In T\textgreater{}1 capsids, the portal is surrounded by MCP hexamers constructed in 6-fold symmetry. Note the tail tube has 3-fold and 6-fold symmetry down the middle of the tube (since it is built of hexameric  rings of tail proteins, so the portal can match up with the tail without a symmetry mismatch. The symmetry mismatch between the portal vertex and the portal itself means that that every one of the portal protein-MCP protein interactions are different, which means getting the details right about the crystallographic structure is very difficult, since averaging of image data at these junctions is not allowed.}
}
\newglossaryentry{tail spike}{
	name={tail spike},
	description={the distal end of the flexible tail in lambda}
}
\newglossaryentry{tail tube}{
	name={tail tube},
	description={a term describing the cylindrical tube located within the sheath; contains the tape measure protein}
}
\newglossaryentry{tape measure protein}{
	name={tape measure protein},
	description={The tape measure protein determines the length of the phage tail}
}
\newglossaryentry{temperate}{
	name={temperate},
	description={a phage capable of the lysogenic or lytic pathway}
}
\newglossaryentry{terminator}{
	name={terminator},
	description={a region of DNA sequence at the end of a gene which ends transcription}
}
\newglossaryentry{titer}{
	name={titer},
	description={In virology, the titer is proportional to the concentration of the virus, usually cited per mL, in a particular sample or preparation. If it is a biological titer, usually determined by plating on a lawn, it is expressed in PFU (plaque-forming units.) For phage T4 under specified conditions with its ideal host, the PFU titer is equal to the virion titer; i.e., there is an EOP (efficiency of plating) \textasciitilde{}1. However, for most viruses and phages, it is difficult to measure the absolute virion titer, especially since many virions that would be visible by microscopy might have damage from the purification or isolation procedure, or even have assembly defects due to the intracellular morphogenesis pathway, and thus not be able to infect successfully. Titers can also be determined by microscopy and other physical means.}
}
\newglossaryentry{transcription factor}{
	name={transcription factor},
	description={In general, transcription factors are DNA-binding proteins that positively affect the initiation of transcription by RNA polymerase at a nearby promoter.}
}
\newglossaryentry{triggering}{
	name={triggering},
	description={the term triggering refers to the still mysterious event in which phage holins suddenly aggregate in two dimensions and soon thereafter form a hole or pinhole.  It is always associated with loss of the PMF.  It is thought to occur when the holin accumulates to a critical two-dimensional concentration.  The critical concentration model is derived from the studies of bacteriorhodopsin (BR) by Khorana and his colleagues.  BR  accumulates in the membrane as a monomer until it reaches a critical concentration, after which the famous purple membrane array is assembled and continues to grow as new BR is produced.}
}
\newglossaryentry{tripartite operators}{
	name={tripartite operators},
	description={a tripartite operator consists of three sub-operators, each with different binding requirements}
}
\newglossaryentry{turbidity}{
	name={turbidity},
	description={a measure of suspended cells in a lysate}
}
\newglossaryentry{tyrosine recombinase}{
	name={tyrosine recombinase},
	description={the enzyme required for site-specific recombination of phage DNA}
}
\newglossaryentry{vRNAP}{
	name={vRNAP},
	description={virion RNA polymerase}
}
\newglossaryentry{vir}{
	name={\textit{vir}},
	description={refers to vir mutants}
}
\newglossaryentry{virulent}{
	name={virulent},
	description={a phage only capable of replicating using the lytic pathway}
}
